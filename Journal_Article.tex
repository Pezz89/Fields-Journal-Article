
\documentclass{scrartcl}
\usepackage{enumitem}
\usepackage[british]{babel}
\usepackage[style=apa, backend=biber]{biblatex}
\DeclareLanguageMapping{british}{british-apa}
\usepackage{url}
\usepackage{float}
\restylefloat{table}
\usepackage{perpage}
\MakePerPage{footnote}
\usepackage{abstract}
\usepackage{graphicx}
% Create hyperlinks in bibliography
\usepackage{hyperref}

\renewcommand{\familydefault}{\sfdefault}
\usepackage{fontspec}
\setmainfont{Arial}

\usepackage{blindtext}
\setkomafont{disposition}{\normalfont\fontsize{12}{17}\bfseries}
\setkomafont{section}{\normalfont\fontsize{12}{17}\bfseries}
\setkomafont{subsection}{\normalfont\fontsize{12}{17}\bfseries\itshape}

\usepackage{titlesec} 

\graphicspath{{./resources/}}
\addbibresource{~/PerryPerrySource/LaTeX/FYP_Bibliography.bib}


\usepackage{etoolbox}
\makeatletter
\expandafter\patchcmd\csname\string\maketitle\endcsname
  {\vskip\z@\@plus3fill}
  {\vskip\z@\@plus2fill\box\abstractbox\vskip\z@\@plus1fill}
  {}{}
\makeatother

\DeclareCiteCommand{\citeyearpar}
    {}
    {\mkbibparens{\bibhyperref{\printdate}}}
    {\multicitedelim}
    {}

\begin{document}
    \title{Descriptor Driven Concatenative Synthesis Tool}
    % \subtitle{\LARGE{Abstract Draft}}
    \author{Sam Perry}

    \maketitle


    \begin{abstract} 
    A command-line tool is proposed for the exploration of a new form of audio
    synthesis known as ``concatenative-synthesis'': A form of synthesis that uses
    perceptual audio analyses to arrange small segments of audio based on their
    characteristics.  The tool is designed to synthesise representations of an
    input sound using a database of source sounds. This involves the
    segmentation and analysis of both the input sound and database, matching of
    input segments to their closest segment from the database, and the
    re-synthesis of the closest matches from the database to produce the final
    result.\\

    The aim was to produce a tool capable of generating high quality sonic
    representations of an input, and to present a variety of examples that
    demonstrated the breadth of possibilities that this style of synthesis has
    to offer. There are a number of other projects that use this form of
    synthesis, however this project aims primarily to explore the further
    potential offered through the offline processing of large databases, of
    which considerably less research exists.\\

    Results demonstrate the wide variety of sounds that can be produced using
    this method of synthesis. A number of technical issues are outlined that
    impeded the overall quality of results and efficiency of the software.
    However, the project clearly demonstrates the strong potential for this
    type synthesis to be used for creative purposes.
    \end{abstract}

    \section*{Background}
    The concept of constructing a new sound by arranging collections of smaller
    sounds has gained popularity in the past 30 years through the introduction
    of ``Granular Synthesis''. Granular synthesis works on the theory that any
    sound can be described through the arrangement of smaller samples (reffered
    to as ``grains''). This representation of sound allows for the temporal
    decomposition and re-arranging of real-world samples, with the potential to
    create new ``complex, dynamically-evolving
    sounds.''~\parencite[p.1]{itgs1988cr}\\

    Concatenative synthesis is a form of synthesis that has developed
    significantly over the past 15 years, driven by recent advancements in
    technology. Key advancements have been in easy access to large databases of
    audio and the development of methods for extracting useful information from
    these databases automatically~\parencite[p.1]{schwarz2006cstey}.
    Concatenative synthesis utilises these technologies to provide a
    content-based extension to granular
    synthesis~\parencite[p.102]{schwarz2007cbcs}; by analysing a database of
    source grains, grains can be differentiated based on their charcteristics.
    These charachteristics can then be used for grain selection in the process
    of synthesizing the output.

    \subsection*{Related Works}
    A number of programs utilize concatenative synthesis to achieve various
    goals. The process has been used for applications in areas such as: 
    \begin{itemize}
        \item Speech Synthesis (Talkapillar)
        \item Creative exploration of databases in a live performance context
            (CCCombine, Riding the Waves)
        \item Musical Instrument Synthesis (Synful)
        \item Musical Sound Synthesis (CataRT, Catapillar)
    \end{itemize}

    The wide range of applications demonstrates the versatility of this
    synthesis technique. It differs from traditional synthesis methods through
    the use of real recorded samples. By transforming samples that have been
    directly recorded from a source, the subtle nuances of the sources sound
    are preserved. These would be difficult to reproduce using other synthetic
    methods for modeling an instrument.~\parencite[p.24]{mrks2009csrs}

    This is particularly important in speech Synthesis
    
    Progress has also been made in instrument Synthesis

    There has also been considerable work on musical sound synthesis, where the
    objective is not to emulate any real sound, but to explore the
    possibilities for synthesizing new abstract sounds for creative purposes.
    Perhaps the most advanced project in this area is CataRT;
    
    \section*{Concatenator Program Design and Implementation}
    \subsection*{Framework Design}
    \subsection*{Descriptor Implementation}
    \subsection*{Matching Algorithms}
    \subsection*{Synthesis and Transformations}
    \subsection*{Command line Interface}

    \section*{Results and Evaluation}

    \section*{Research Limitations/Potential Improvments}
    Given the limited time frame and complexity of modern approaches to this
    form of synthesis, only a basic implementation was possible.
    \section*{Conclusion}
\end{document}
