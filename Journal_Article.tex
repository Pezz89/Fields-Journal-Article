\documentclass{scrartcl}
\usepackage{enumitem}
\usepackage[british]{babel}
\usepackage[style=apa, backend=biber]{biblatex}
\DeclareLanguageMapping{british}{british-apa}
\usepackage{url}
\usepackage{float}
\restylefloat{table}
\usepackage{perpage}
\MakePerPage{footnote}
\usepackage{abstract}
\usepackage{graphicx}
% Create hyperlinks in bibliography
\usepackage{hyperref}

\renewcommand{\familydefault}{\sfdefault}
\usepackage{fontspec}
\setmainfont{Arial}

\usepackage{blindtext}
\setkomafont{disposition}{\normalfont\fontsize{12}{17}\bfseries}
\setkomafont{section}{\normalfont\fontsize{12}{17}\bfseries}
\setkomafont{subsection}{\normalfont\fontsize{12}{17}\bfseries\itshape}
\setkomafont{subsubsection}{\normalfont\fontsize{12}{17}\itshape}

\graphicspath{{./resources/}}
\addbibresource{~/PerryPerrySource/LaTeX/library.bib}

\usepackage{etoolbox}
\makeatletter
\expandafter\patchcmd\csname\string\maketitle\endcsname
  {\vskip\z@\@plus3fill}
  {\vskip\z@\@plus2fill\box\abstractbox\vskip\z@\@plus1fill}
  {}{}
\makeatother

\DeclareCiteCommand{\citeyearpar}
    {}
    {\mkbibparens{\bibhyperref{\printdate}}}
    {\multicitedelim}
    {}

\begin{document}
    \title{Descriptor Driven Concatenative Synthesis Tool}
    % \subtitle{\LARGE{Abstract Draft}}
    \author{Sam Perry}

    \maketitle

    \begin{abstract} 
    A command-line tool is proposed for the exploration of a new form of audio
    synthesis known as ``concatenative-synthesis'' (CS): A form of synthesis that uses
    perceptual audio analyses to arrange small segments of audio based on their
    characteristics.  The tool is designed to synthesise representations of an
    input sound using a database of source sounds. This involves the
    segmentation and analysis of both the input sound and database, matching of
    input segments to their closest segment from the database, and the
    re-synthesis of the closest matches from the database to produce the final
    result.\\

    The aim was to produce a tool capable of generating high quality sonic
    representations of an input, and to present a variety of examples that
    demonstrated the breadth of possibilities that this style of synthesis has
    to offer. There are a number of other projects that use this form of
    synthesis, however this project aims primarily to explore the further
    potential offered through the offline processing of large databases, of
    which considerably less research exists.\\

    Results demonstrate the wide variety of sounds that can be produced using
    this method of synthesis. A number of technical issues are outlined that
    impeded the overall quality of results and efficiency of the software.
    However, the project clearly demonstrates the strong potential for this
    type synthesis to be used for creative purposes.
    \end{abstract}

    \section*{Background}
    The concept of constructing a new sound by arranging collections of smaller
    sounds has gained popularity in the past 30 years through the introduction
    of ``Granular Synthesis''. Granular synthesis works on the theory that any
    sound can be described through the arrangement of smaller samples (reffered
    to as ``grains''). This representation of sound allows for the temporal
    decomposition and re-arranging of real-world samples, with the potential to
    create new ``complex, dynamically-evolving
    sounds.''~\parencite[p.1]{Roads1988}\\

    Concatenative synthesis is a form of synthesis that has developed
    significantly over the past 15 years, driven by recent advancements in
    technology. Key advancements have been in easy access to large databases of
    audio and the development of methods for extracting useful information from
    these databases automatically~\parencite[p.1]{Schwarz2006}.  CS utilises
    these technologies to provide a content-based extension to granular
    synthesis; by analysing a database of source grains, grains can be
    differentiated based on their charcteristics.  These charachteristics can
    then be used for grain selection in the process of synthesizing output for
    a wide range of applications~\parencite[p.102]{Schwarz2007}.

    \subsection*{Related Works}
    A number of programs utilize CS to achieve various goals. The process has
    been used for applications in areas such as Speech Synthesis, Instrument
    synthesis and for applications in creative sound design.\\
    The wide range of applications demonstrates the versatility of this
    synthesis technique. It differs from traditional synthesis methods through
    the use of real recorded samples, as opposed to traditional methods that
    focus on defining sets of rules for emulating real sounds. By transforming
    samples that have been directly recorded from a source, the subtle nuances
    of the source's sound are preserved. These would be difficult to reproduce
    using other synthetic methods for modeling an
    instrument~\parencite[p.24]{Maestre2009a}.

    \subsubsection*{Speech Synthesis}
    Creating a natural and intelligible realisation is an important factor when
    developing a speech synthesis system.*add part about continuity here* The
    Talkapillar project is one such example of how highly convincing results
    are possible with CS. Through careful analysis of a vocal database, the
    project aims to impose the qualities of the database voice on an input
    voice. This would result in the words of the input speaker being
    transformed to appear as if they were spoken by the voice in the
    database.~\parencite{Hueber}
    
    \subsubsection*{Instrument Synthesis}
    Progress has also been made in improving the quality of instrument
    synthesis. As with speech synthesis, the use of samples directly allows for
    natural sounding results, which provides a method for reproducing real
    instruments convincingly.\\
    Another important aspect of instrument synthesis is that of performer
    expression. The reproduction of performance qualities such as dynamics,
    timbre and timing are essential when emulating a real instrument and CS has
    been used to effectively reproduce these aspects. This is achieved through
    splicing of grains based on their expressive characteristics to form
    musical phrases.  For example, just as a violinist might transition
    seamlessly from one articulation to the next, the CS software will join
    grains to produce the varyation in articulations. This contrasts the
    traditional approach to sampling, where samples are played in isolation,
    resulting in a discontinuity between adjacent samples.  The comercial
    software synthesizer ``Synful'' (\url{www.synful.com}) successfully
    demonstrates the use of CS to produce highly convincing recreations of
    orchestral instrument performances in real-time.
    ~\parencite[p.82]{Lindemann2007}.

    \subsubsection*{Creative Sound Design}
    The flexibilty of CS allows for creativity in a broader context than simply
    emulating real-world instruments and speech. It can also be used as a tool
    to explore the possibilities for synthesizing new abstract sounds for
    creative purposes.\\
    A prominent project in this area of CS is IRCAM's CataRT project *needs
    reference*. The project focuses on the playback of source grains based on
    their proximity to a target in multi-dimensional descriptor space. 
    By providing a target point in the descriptor space, the user is able to
    navigate the database, playing selections of samples that are nearest to
    the target. This allows the user to explore the database intuitively
    through a graphic user interface, selecting a point in 2-dimensional space
    with the mouse. Grains are then played back in real-time to create an
    ``audio mosaic''.\\
    Alternatively, target audio can be provided and analysed to create a target
    location based on it's location in the descriptor space.  Tremblay and
    Schwarz's~\citeyearpar{Tremblay2010} use of CataRT to explore
    electroacoustic sample banks demonstrates the creative potential of this
    method. CS is used in this context as a means for synthesizing matches in a
    corpus database to real-time input from an electric bass.  Significance is
    placed on linking the playback of grains to the expressivity of the
    performer. The use of perceptualy based audio descriptors to match the
    source to the target allows the performer to navigate the database
    naturally based on factors such as the pitch and timbre of the bass
    guitar. The result is a performance that mixes characteristics of both the
    bass guitar output and the qualities of the corpus database to create a
    hybrid of the two.
    
    \section*{Concatenator Program Design and Implementation}
    Aims:
    instrument resynthesis onto a pre-existing source sound, rather than from scratch onto things like midi notes.
    Offline processing to allow for large databases to be used - disadvantage: loss of feedback between performer and system, as described in PA's paper.
    advantage: Real-time approach results in reduced continuity of grains
    \subsection*{Framework Design}
    \subsection*{Descriptor Implementation}

    \subsection*{Matching Algorithms}
    \subsection*{Synthesis and Transformations}

    \subsection*{Command line Interface}
    High quantity of parameters is very time consuming ~\parencite{Petrushin2007} 

    \section*{Results and Evaluation}

    \section*{Research Limitations/Potential Development}
    Given the limited time frame and complexity of modern approaches to this
    form of synthesis, only a basic implementation was possible.

    Spectral matching~\parencite{Hoffman2009} 
    Use of RPM?~\parencite[p.82]{Lindemann2007}
    Viterbi path search~\parencite[p.1]{Schwarz2006a}

    \section*{Conclusion}

    \printbibliography
\end{document}
